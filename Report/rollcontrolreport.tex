\documentclass[12pt]{article}
\usepackage{amsmath}
\usepackage{amssymb}
\usepackage{graphicx}
\usepackage{tabulary}


\begin{document}

\title{Roll Control for LV2.3}
\author{William Harrington\\ %Nate: if you edit this document, please include your name next to mine
\textit{Portland State Aerospace Society}} %if necessary, replace with your course title
 
\maketitle

\begin{description}
	\item[Introduction] \hfill \\
		The roll control system for the current generation of launch vehicle (2.3) has more or less not been functioning properly since Launch 8. (Nate: explain here some of the problems with previous incarnations of the roll control system). In June of 2014, another attempt at designing a roll control system was initiated. The purpose of this report is to document the process of re-developing the roll control system.
		
\end{description}
 
\begin{description} 
	\item[State space model] \hfill \\
		In this section we set out to develop a state space model for the roll dynamics of the rocket. \\

		Equations approximating the motion of the rocket are derived from free-body diagrams (FBD) and kinetic diagrams (KD), as taught in a mid level mechanical engineering dynamics course. Developing equations in this manner is known as developing from first principles because basic Newtonian principles $F=ma$ are used to derive them. \cite{PSAS} The diagrams are shown below:\\
		\begin{itemize}
			\item[] \includegraphics[scale=.5]{350x250-RollFBD.png} \\
			
				\textit{From this diagram we get the following differential equation}\\
				\textit{$I_{zz}\ddot{\theta} = l_{f}u - K_{d}\dot{\theta}$} \cite{PSAS} \\
				
			\newpage
			\item[] Modern control theory is founded on the state space approach. Therefore, we will review and discuss important terminology that pertains to developing our state space model. \\
			 
			 The \textbf{state} of a dynamic system is the smallest set of variables (called \textit{state variables}) such that knowledge of these variables at $t = t_{0}$, together with knowledge of the input for $ t \geq t_{0}$, completely determines the behavior of the system for any time $t \geq t_{0}$. \cite{MCE}\\
			
			The \textbf{state variables} of a dynamic system are the variables making up the smallest set of variables that determine the state of the dynamic system. If at least \textit{n} variables \textit{$x_{1}, x_{2},..., x_{n}$} are needed to completely describe the behavior of a dynamic system (so that once the input is given for $t \geq t_{0}$ and the initial state at $t = t_{0}$ is specified, the future state of the system is completely determined), then such \textit{n} variables are a set of state variables. \cite{MCE}\\
			
			If \textit{n} state variables are needed to completely describe the behavior of a given system, then these \textit{n} state variables can be considered the \textit{n} components of a veto \textbf{x}. Such a vector is called a \textbf{state vector}. A state vector is thus a vector that determines uniquely the system \textbf{x(t)} for any time $t \geq t_{0}$, once the state at $t = t_{0}$ is given and the input \textbf{u(t)} for $t \geq t_{0}$ is specified. \cite{MCE} \\
			
			The \textit{n}-dimensional space whose coordinate axes consist of the \textit{$x_{1}$} axis, \textit{$x_{2}$} axis,..., \textit{$x_{n}$} axis, where \textit{$x_{1}, x_{2},..., x_{n}$} are state variables, is called a \textbf{state space}. Any state can be represented by a point in the state space. \cite{MCE} \\
			
			In state-space analysis we are concerned with three types of variables: input variables, output variables, and state variables.
			
			\item[]

				\textit{Blahblah} \\
				
		\end{itemize}
		
\end{description}
 
\newpage
 \begin{thebibliography}{9}
 \bibitem{PSAS}
 Portland State Aerospace Society, Roll Control Wiki
 \texttt{http://psas.pdx.edu/rollcontrol/}
 \bibitem{MCE}
 Katsuhiko Ogata
 \textit{Modern Control Engineering, Fifth Edition}
 Prentice Hall, Boston, MA, 2010
 \end{thebibliography}
\end{document}